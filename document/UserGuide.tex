%%%%%%%%%%%%%%%%%%%%%%%%%%%%%%%%%%%%%%%%%%%%%%%%%%
% Simple Drawing FrameWork User Guide
% wrtten by Nakaue Yoshito
% 2022/03/06
%%%%%%%%%%%%%%%%%%%%%%%%%%%%%%%%%%%%%%%%%%%%%%%%%%

\documentclass[a4paper, 11pt, oneside, onecolumn, openany]{jsarticle}

\usepackage[dvipdfmx]{graphicx}
\usepackage{here}
\usepackage{tabto}

\begin{document}
\NumTabs{20}

\newcommand{\function}[3]{
  \noindent
  \tab \texttt{#1} \\
  \tab \underline{引数} \tab #2 \\
  \tab \underline{返り値} \tab #3 \\\par
}

%%%%%%%%%%%%%%%%%%%%%%%%%%%%%%%%%%%%%%%%%%%%%%%%%%
% Title
%%%%%%%%%%%%%%%%%%%%%%%%%%%%%%%%%%%%%%%%%%%%%%%%%%
\title{\vspace{-3cm}Simple Drawing FrameWork}
\author{Y.Nakaue}
\maketitle


%%%%%%%%%%%%%%%%%%%%%%%%%%%%%%%%%%%%%%%%%%%%%%%%%%
% プログラムの開始と終了
%%%%%%%%%%%%%%%%%%%%%%%%%%%%%%%%%%%%%%%%%%%%%%%%%%
\section{プログラムの開始と終了}
\subsection{ライブラリ機能の初期化}
\function{void init()}{無し}{無し}

\subsection{ライブラリ機能の終了処理}
\function{void quit()}{無し}{無し}


%%%%%%%%%%%%%%%%%%%%%%%%%%%%%%%%%%%%%%%%%%%%%%%%%%
% メインループ
%%%%%%%%%%%%%%%%%%%%%%%%%%%%%%%%%%%%%%%%%%%%%%%%%%
\section{メインループ}
\subsection{描画内容の更新}
\function{bool System::update}{無し}{メインループの更新可否}
描画内容を最新の状態に更新する。常に true を返す。


%%%%%%%%%%%%%%%%%%%%%%%%%%%%%%%%%%%%%%%%%%%%%%%%%%
% 描画のための設定
%%%%%%%%%%%%%%%%%%%%%%%%%%%%%%%%%%%%%%%%%%%%%%%%%%
\section{描画のための設定}
\subsection{ウィンドウの作成}
\function{int32\_t openWindow(uint32\_t width, uint32\_t height)}{width: 横幅,height: 高さ}{作成したウィンドウID}
描画を行うためのウィンドウを作成し,画面前面に表示する.引数には作成するウィンドウの横幅と高さを指定する.1つのプログラムの中で複数のウィンドウを作成することが可能で,返り値として作成したウィンドウに割り当てられたウィンドウIDを返す.ウィンドウIDは,作成した順に0からの連番で整数値が返される.

\subsection{ウィンドウを閉じる}
\function{void closeWindow(int32\_t win\_id)}{win\_id: 閉じるウィンドウのID}{無し}


%%%%%%%%%%%%%%%%%%%%%%%%%%%%%%%%%%%%%%%%%%%%%%%%%%
% 主な描画関数
%%%%%%%%%%%%%%%%%%%%%%%%%%%%%%%%%%%%%%%%%%%%%%%%%%
\section{主な描画関数}
\subsection{背景色を変更}
\function{void setBackground(Color color, int32\_t win\_id = 0)}{color: 背景色,win\_id: 背景色を設定するウィンドウのID}{無し}
背景色を引数で指定子た色に設定する。新たな背景色は、この関数によって設定した次のフレーム(System::update()が呼び出されたタイミング)で反映される。


\end{document}
