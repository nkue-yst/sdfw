%%%%%%%%%%%%%%%%%%%%%%%%%%%%%%%%%%%%%%%%%%%%%%%%%%
% Simple Drawing FrameWork User Guide
% wrtten by Nakaue Yoshito
% 2022/03/06
%%%%%%%%%%%%%%%%%%%%%%%%%%%%%%%%%%%%%%%%%%%%%%%%%%

\documentclass[a4paper, 11pt, oneside, onecolumn, openany]{jsarticle}

\usepackage[dvipdfmx]{graphicx}
\usepackage{here}
\usepackage{tabto}

\begin{document}
\NumTabs{20}

%%%%%%%%%%%%%%%%%%%%%%%%%%%%%%%%%%%%%%%%%%%%%%%%%%
% Title
%%%%%%%%%%%%%%%%%%%%%%%%%%%%%%%%%%%%%%%%%%%%%%%%%%
\title{\vspace{-3cm}Simple Drawing FrameWork}
\author{Y.Nakaue}
\maketitle


%%%%%%%%%%%%%%%%%%%%%%%%%%%%%%%%%%%%%%%%%%%%%%%%%%
% プログラムの開始と終了
%%%%%%%%%%%%%%%%%%%%%%%%%%%%%%%%%%%%%%%%%%%%%%%%%%
\section{プログラムの開始と終了}
\subsection{ライブラリ機能の初期化}
\noindent
\tab void init() \\
\tab \underline{引数} \tab 無し \\
\tab \underline{返り値} \tab 無し \\

\subsection{ライブラリ機能の終了処理}
\tab void quit() \\
\tab \underline{引数} \tab 無し \\
\tab \underline{返り値} \tab 無し \\


%%%%%%%%%%%%%%%%%%%%%%%%%%%%%%%%%%%%%%%%%%%%%%%%%%
% 描画のための設定
%%%%%%%%%%%%%%%%%%%%%%%%%%%%%%%%%%%%%%%%%%%%%%%%%%
\section{描画のための設定}
\subsection{ウィンドウの作成}
\noindent
\tab int32\_t openWindow(uint32\_t width, uint32\_t height) \\
\tab \underline{引数} \tab width: 横幅,height: 高さ \\
\tab \underline{返り値} \tab 作成したウィンドウID \\

描画を行うためのウィンドウを作成し,画面前面に表示する.引数には作成するウィンドウの横幅と高さを指定する.1つのプログラムの中で複数のウィンドウを作成することが可能で,返り値として作成したウィンドウに割り当てられたウィンドウIDを返す.ウィンドウIDは,作成した順に0からの連番で整数値が返される.


\end{document}
