%%%%%%%%%%%%%%%%%%%%%%%%%%%%%%%%%%%%%%%%%%%%%%%%%%
% Simple Drawing FrameWork User Guide
% wrtten by Nakaue Yoshito
% 2022/03/06
%%%%%%%%%%%%%%%%%%%%%%%%%%%%%%%%%%%%%%%%%%%%%%%%%%

\documentclass[a4paper, 11pt, oneside, onecolumn, openany]{jsarticle}

\usepackage[dvipdfmx]{graphicx}
\usepackage{here}
\usepackage{tabto}
\usepackage{listings}
\usepackage{jlisting}

\begin{document}
\NumTabs{20}

\newcommand{\function}[3]{
  \noindent
  \tab \texttt{#1} \\
  \tab \underline{引数} \tab #2 \\
  \tab \underline{返り値} \tab #3 \\\par
}

%%%%%%%%%%%%%%%%%%%%%%%%%%%%%%%%%%%%%%%%%%%%%%%%%%
% Setting for source code
%%%%%%%%%%%%%%%%%%%%%%%%%%%%%%%%%%%%%%%%%%%%%%%%%%
\lstset{
  basicstyle={\ttfamily},
  identifierstyle={\small},
  commentstyle={\smallitshape},
  keywordstyle={\small\bfseries},
  ndkeywordstyle={\small},
  stringstyle={\small\ttfamily},
  frame={tb},
  breaklines=true,
  columns=[l]{fullflexible},
  numbers=left,
  xrightmargin=0zw,
  xleftmargin=3zw,
  numberstyle={\scriptsize},
  stepnumber=1,
  numbersep=1zw,
  lineskip=-0.5ex
}

%%%%%%%%%%%%%%%%%%%%%%%%%%%%%%%%%%%%%%%%%%%%%%%%%%
% Title
%%%%%%%%%%%%%%%%%%%%%%%%%%%%%%%%%%%%%%%%%%%%%%%%%%
\title{\vspace{-3cm}Simple Drawing FrameWork}
\author{Y.Nakaue}
\maketitle


%%%%%%%%%%%%%%%%%%%%%%%%%%%%%%%%%%%%%%%%%%%%%%%%%%
% プログラムの開始と終了
%%%%%%%%%%%%%%%%%%%%%%%%%%%%%%%%%%%%%%%%%%%%%%%%%%
\section{プログラムの開始と終了}
\subsection{ライブラリ機能の初期化}
\function{void init()}{無し}{無し}

\subsection{ライブラリ機能の終了処理}
\function{void quit()}{無し}{無し}


%%%%%%%%%%%%%%%%%%%%%%%%%%%%%%%%%%%%%%%%%%%%%%%%%%
% メインループ
%%%%%%%%%%%%%%%%%%%%%%%%%%%%%%%%%%%%%%%%%%%%%%%%%%
\section{メインループ}
\subsection{描画内容の更新}
\function{bool System::update}{無し}{メインループの更新可否}
描画内容を最新の状態に更新する.常にtrueを返す.\\


%%%%%%%%%%%%%%%%%%%%%%%%%%%%%%%%%%%%%%%%%%%%%%%%%%
% 描画のための設定
%%%%%%%%%%%%%%%%%%%%%%%%%%%%%%%%%%%%%%%%%%%%%%%%%%
\section{描画のための設定}
\subsection{ウィンドウの作成}
\function{int32\_t openWindow(uint32\_t width, uint32\_t height)}{width: 横幅,height: 高さ}{作成したウィンドウID}
描画を行うためのウィンドウを作成し,画面前面に表示する.引数には作成するウィンドウの横幅と高さを指定する.1つのプログラムの中で複数のウィンドウを作成することが可能で,返り値として作成したウィンドウに割り当てられたウィンドウIDを返す.ウィンドウIDは,作成した順に0からの連番で整数値が返される.

\subsection{ウィンドウを閉じる}
\function{void closeWindow(int32\_t win\_id)}{win\_id: 閉じるウィンドウのID}{無し}


%%%%%%%%%%%%%%%%%%%%%%%%%%%%%%%%%%%%%%%%%%%%%%%%%%
% 主な描画関数
%%%%%%%%%%%%%%%%%%%%%%%%%%%%%%%%%%%%%%%%%%%%%%%%%%
\section{主な描画関数}
\subsection{背景色を変更}
\function{void setBackground(Color color, int32\_t win\_id = 0)}{color: 背景色,win\_id: 背景色を設定するウィンドウのID}{無し}
背景色を引数で指定した色に設定する.新たな背景色は,この関数によって設定した次のフレーム(System::update()が呼び出されたタイミング)で反映される.

\subsection{文字列}
\function{void print(std::string str, int32\_t win\_id = 0)}{str: 文字列データ,win\_id: 描画先ウィンドウID}{無し}
引数に指定した文字列を描画する.第2引数に描画を行うウィンドウIDを指定でき,デフォルトではメインウィンドウへの描画を行う.1フレーム内で複数回この関数を呼び出すと自動的に改行され,2回目以降の呼び出し時には次の行に出力される.

\subsection{図形}
図形各種は、基底クラスであるShapeクラスを継承したクラスのオブジェクト(インスタンス)を作成し、そのメンバ関数であるdraw()を呼び出すことで描画を行うことができる。

\subsubsection{線分}
線分の描画を行うにはLineを作成し、そのdraw()を呼び出す。
\begin{lstlisting}[caption=使用例, keepspaces=true]
  #include <sdfw.h>

  using namespace sdfw;
  
  int main()
  {
      init();
      int32_t win = openWindow(1280, 720);
  
      while (System::update())
      {
          // (500, 500)から(600, 600)までの線分を描画する
          Line(500, 500, 600, 600).draw();
      }
  
      closeWindow(win);
      quit();
  }
\end{lstlisting}


%%%%%%%%%%%%%%%%%%%%%%%%%%%%%%%%%%%%%%%%%%%%%%%%%%
% イベントに関する機能
%%%%%%%%%%%%%%%%%%%%%%%%%%%%%%%%%%%%%%%%%%%%%%%%%%
\section{イベントに関する機能}
以下の機能はSystem::update()が呼び出されたタイミングで状態が更新されているため,少し古い情報が得られる場合がある.
\subsection{マウス入力}
\subsubsection{マウスカーソル座標の取得}
\function{Point pos()}{無し}{現在のマウスカーソル座標}
現在のマウスカーソルの座標値をPoint型で取得できる。また、x座標・y座標はそれぞれメンバ変数x・yから取得できる。
\begin{lstlisting}[caption=使用例, label=macro-GetMouseButtonState, keepspaces=true]
#include <sdfw.h>
#include <string>

using namespace sdfw;

int main()
{
    init();

    int32_t win = openWindow(1280, 720);

    std::string str;
    while (System::update())
    {
        // マウスカーソルの座標をテキスト出力
        str = "X: " + std::to_string(Mouse::pos().x) + ", Y: " + std::to_string(Mouse::pos().y);
        print(str);
    }

    closeWindow(win);
    quit();
}
\end{lstlisting}

\subsubsection{マウスボタン入力状態の取得(押されているかを取得)}
\function{bool pressed(int8\_t button)}{button: マウスボタン(マクロ)}{指定したマウスボタンが押されているか}
入力状態を得たいマウスボタンの指定には,マクロによるビットマスクを使用する.ここで使用するマクロは以下の表\ref{macro-MouseButton}のように定義されている.また,これらをビットORで結合して引数に指定することで,複数のボタンが同時に入力されているかを得ることもできる.
\begin{table}[H]
  \caption{マウスボタンを表すマクロ名}
  \label{macro-MouseButton}
  \centering
  \begin{tabular}{cc}
    \hline
    ボタン & マクロ名 \\
    \hline \hline
    マウス左ボタン & LEFT \\
    マウス中ボタン & MIDDLE \\
    マウス右ボタン & RIGHT \\
    \hline
  \end{tabular}
\end{table}

\begin{lstlisting}[caption=使用例, label=macro-GetMouseButtonState, keepspaces=true]
  // マウス左ボタンが押されているかを取得
  sdfw::Mouse::pressed(LEFT);

  // マウス左・右ボタンが両方押されているかを取得
  sdfw::Mouse::pressed(LEFT | RIGHT);
\end{lstlisting}


%%%%%%%%%%%%%%%%%%%%%%%%%%%%%%%%%%%%%%%%%%%%%%%%%%
% タイマ関連機能
%%%%%%%%%%%%%%%%%%%%%%%%%%%%%%%%%%%%%%%%%%%%%%%%%%
\section{タイマ関連機能}
\subsection{SDFWが初期化されてから経過したフレーム数を取得する}
\function{uint32\_t Time::getTicks()}{無し}{現在のフレーム数}
SDFWを初期化してからのフレーム数を返す.フレーム数と時間との関係は,プログラム実行時のフレームレートに依存する.

\subsection{SDFWが初期化されてから経過した時間(ミリ秒)を取得する}
\function{uint32\_t Time::getMillisec()}{無し}{現在の経過時間(ミリ秒)}
SDFWを初期化してからの経過時間をミリ秒で返す.

\subsection{現在フレームまでの平均FPS値を取得する}
\function{float getAverageFPS()}{無し}{平均FPS}
現在のフレームまでの平均FPS値を取得する。\par
※この値の精度は低い。


\end{document}
